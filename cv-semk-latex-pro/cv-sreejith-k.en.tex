\documentclass{resume}

\renewcommand{\categoryfont}{\sc}

%
% set the space used for category titles here:
% use the same value for oddsidemargin and marginparwidth [the latter 
% 		will be reset to account for marginparsep]
% 
\setlength{\oddsidemargin}{1in}
\setlength{\marginparwidth}{1in}
% 
% calculate other dimensions [textwidth and evensidemargin] 
% in function of oddsidemargin and marginparwidth: 
% would be nicer to put in the class file...
%
\addtolength{\marginparwidth}{-\marginparsep}
 \setlength{\evensidemargin}{\oddsidemargin}
 \setlength{\textwidth}{\paperwidth}
 \addtolength{\textwidth}{-2in}
 \addtolength{\textwidth}{-2\oddsidemargin}
 \addtolength{\textwidth}{\marginparwidth}
 \addtolength{\textwidth}{\marginparsep}
%
%
\setlength{\topmargin}{-0.5in}
%
%
\renewcommand{\labelcitem}{$\diamond$}
\renewcommand{\labelitemi}{$\cdot$}
\newcommand{\first}{$1^{\mbox{\scriptsize st}}$\ }
\newcommand{\second}{$2^{\mbox{\scriptsize nd}}$\ }
\newcommand{\third}{$3^{\mbox{\scriptsize rd}}$\ }

\author{Sreejith Kesavan}
% ------ Address --------------------------------------------------------

\address{Member Technical Staff - II\\
  NetApp India Private Limited\\
  Bangalore, Karnataka\\
  {\em Ph:} (+91) 9940 3821 62}{
  {\em email:} \mbox{\small\tt sreejithemk@gmail.com}\\
  {\em web:} \mbox{\small\tt foobarnbaz.com}\\
  {\em github:} \mbox{\small\tt github.com/semk}\\
  {\em linkedin:} \mbox{\small\tt in.linkedin.com/in/sreejithemk}}

\begin{document}
\maketitle

% ------- Education ---------------------------------------------------

\begin{category}{Education}
\citem{B-Tech in Computer Science \& Engineering} \hfill {\em 2004 -- 2008}\\
University of Calicut, Jyothi Engineering College, Kerala.
\citem{Higher Secondary Education in Computer Science} \hfill {\em 2002 -- 2004}\\
Kshethra Pravesana Memorial Higher Secondary School, Poothotta, Kerala.
\end{category}

% --------- Research ----------------------------------------------------

\begin{category}{Research interests}
\citemnobullet NoSQL databases, application development in Mac OSX \& iOS, 
distributed and concurrent programming with Erlang, Amazon Dynamo architecture.
\end{category}

\begin{category}{Technical Skills}
\citembullet {\it Operating Systems}: GNU/Linux, Mac OSX, Microsoft Windows.
\citembullet {\it Languages}: Python, Objective C, C, Erlang.
\citembullet {\it Databases}: MySQL, SQLite, Hypertable, MongoDB, Riak, Redis.
\citembullet {\it Tools}: Memcached, RabbitMQ, HAProxy, Nginx.
\citembullet {\it Platforms}: Apache Hadoop, Google App Engine, Amazon EC2, Amazon S3.
\citembullet {\it Specialities}: Design and development of scalable web services, Big Data, 
NoSQL databases, virtualization, Infrastructure as a Service, Platform as a Service, 
userspace filesystems, Debian and RPM package management.
\end{category}

% -------- Work experience --------------------------------------------

\begin{category}{Work \\experience}
\citem{Member Technical Staff - II}, {\em NetApp India Private Limited} \hfill {\em Sep 2012 -- present}\\
Working as part of a team responsible for participating in the development, testing and 
debugging of operating systems and file systems that run NetApp storage applications. 
As part of the Research and Development function, the overall focus of the group is on 
competitive market and customer requirements, technology advances, product quality, 
product cost and time-to-market.
\citem{Development Engineer}, {\em K7 Computing Private Limited} \hfill {\em Feb 2009 -- Aug 2012}\\
Actively involved in the design and development of a cloud computing platform that can solve 
the problems of application scalability, availability and fault-tolerance.
\begin{itemize} 
\item Lead a team of 3 developers to build a scalable platform for Python \& PHP.
\item Helped clients to migrate traditional applications to the cloud.
\item Developed core services and Google App Engine compatible apis for a PaaS project.
\item Developed various FUSE based userspace filesystems for linux clusters.
\end{itemize}
\end{category}

% -------- Awards & Honours ---------------------------------------

\begin{category}{Awards \&\\ Honours}
\citem{Excellence in Software Development}, {\em K7 Computing Private Limited}. \hfill {\em 2011}
\citem{Best Project}, {\em Computer Science, Jyothi Engineering College}. \hfill {\em 2008}
\citem{Best Student}, {\em Edakkattuvayal Grama Panchayat, Ernakulam, Kerala.} \hfill {\em 2002}
\end{category}

% -------- Publication --------------------------------------------

\begin{category}{Articles}
\citem{Linux For You}, {\em ``The New Scheduler on the Block, Dedicated to Desktops''}. \hfill {\em Oct 2009}\\
An article about Brain Fuck Scheduler (BFS) for Linux written by famous Linux Kernel hacker Con Kolivas.
\citem{ILUG-Cochin}, {\em ``Custom Kernel Compilation - Ubuntu way''}. \hfill {\em Aug 2008}\\
An article on compiling custom kernel sources on Ubuntu.
\citem{ILUG-Cochin}, {\em ``Remastering Ubuntu''}. \hfill {\em Aug 2008}\\
An article on building custom Ubuntu distributions using Reconstructor.
\end{category}

\newpage

% -------- Opensource -------------------------------------------

\begin{category}{Opensource}
\citemnobullet All opensource projects can be found on the GitHub page. The major ones are
\citembullet \textbf{Cyclozzo OSE}: Opensource Edition of the Cyclozzo Platform as a Service.
\citembullet \textbf{Voldemort}: Voldemort is a blog-aware static site generator using Jinja2 
and Markdown tem- plates. Inspired by Jekyll static site generator written in Ruby.
\citembullet \textbf{Riak Python client}: Various fixes and feature additions to the Riak Python Client.
\end{category}

% -------- Reference --------------------------------------------

\begin{category}{Reference} 
\citemnobullet Available on request.
\end{category}

\end{document}
