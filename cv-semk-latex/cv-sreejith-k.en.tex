\documentclass{resume}

\renewcommand{\categoryfont}{\sc}

%
% set the space used for category titles here:
% use the same value for oddsidemargin and marginparwidth [the latter 
% 		will be reset to account for marginparsep]
% 
\setlength{\oddsidemargin}{1in}
\setlength{\marginparwidth}{1in}
% 
% calculate other dimensions [textwidth and evensidemargin] 
% in function of oddsidemargin and marginparwidth: 
% would be nicer to put in the class file...
%
\addtolength{\marginparwidth}{-\marginparsep}
 \setlength{\evensidemargin}{\oddsidemargin}
 \setlength{\textwidth}{\paperwidth}
 \addtolength{\textwidth}{-2in}
 \addtolength{\textwidth}{-2\oddsidemargin}
 \addtolength{\textwidth}{\marginparwidth}
 \addtolength{\textwidth}{\marginparsep}
%
%
\setlength{\topmargin}{-0.5in}
%
%
\renewcommand{\labelcitem}{$\diamond$}
\renewcommand{\labelitemi}{$\cdot$}
\newcommand{\first}{$1^{\mbox{\scriptsize st}}$\ }
\newcommand{\second}{$2^{\mbox{\scriptsize nd}}$\ }
\newcommand{\third}{$3^{\mbox{\scriptsize rd}}$\ }

\author{Sreejith Kesavan}
% ------ Address --------------------------------------------------------

\address{Member Technical Staff - II\\
  NetApp Systems Private Limited\\
  Bangalore, Karnataka\\
  {\em Ph:} (+91)-99-8684-9928}{
  {\em email:} \mbox{\small\tt sreejithemk@gmail.com}\\
  {\em web:} \mbox{\small\tt foobarnbaz.com}\\
  {\em github:} \mbox{\small\tt github.com/semk}\\
  {\em linkedin:} \mbox{\small\tt linkedin.com/in/sreejithemk}}

\begin{document}
\maketitle

% ------- Education ---------------------------------------------------

\begin{category}{Education}
  \citem{B-Tech in Computer Science \& Engineering} \hfill \textbf{2004 -- 2008}
  \citemnobullet University of Calicut, Jyothi Engineering College, Kerala.
  \citem{Higher Secondary Education in Computer Science} \hfill \textbf{2002 -- 2004}
  \citemnobullet Kshethra Pravesana Memorial Higher Secondary School, Poothotta, Kerala.
\end{category}

% --------- Research ----------------------------------------------------

\begin{category}{Research interests}
  \citemnobullet NoSQL databases, application development for Mac OSX \& iOS, Green Computing with 
  ARM based servers, distributed and concurrent programming with Erlang, Amazon Dynamo architecture, 
  development on Raspberry Pi.
\end{category}

\begin{category}{Technical Skills}
  \citembullet {\it Operating Systems}: GNU/Linux, Mac OSX, Microsoft Windows.
  \citembullet {\it Programming Languages}: Python, Objective C, C, Erlang.
  \citembullet {\it Frameworks}: Django, Pylons, Tornado, Flask, Bottle, webpy.
  \citembullet {\it Databases}: MySQL, SQLite, Hypertable, Riak, Redis.
  \citembullet {\it Tools}: Memcached, RabbitMQ, HAProxy, Nginx.
  \citembullet {\it Version Control}: Git, Subversion, Mercurial, Perforce.
  \citembullet {\it Cloud Platforms}: Apache Hadoop, Google App Engine, Amazon EC2, Amazon S3.
  \citembullet {\it Specialities}: Python, Cloud Computing, Scalable design, Big Data, Commandline 
  tools, NoSQL databases, Virtualization, API design, Web services, Automation \& deployment, 
  Desktop applications, Shell scripting, Userspace filesystems, Microframeworks, System integration, 
  Package management.
\end{category}

% -------- Work experience --------------------------------------------

\begin{category}{Work \\experience}
  \citem{NetApp Systems Private Limited, Bangalore}
  \citemnobullet \textbf{Member Technical Staff - II} \hfill \textbf{Sep 2012 -- present}
  \citemnobullet Working as part of a team responsible for participating in the development, 
  testing and debugging of operating systems and file systems that run NetApp storage applications. 
  As part of the Research and Development function, the overall focus of the group is on 
  competitive market and customer requirements, technology advances, product quality, 
  product cost and time-to-market.
  \begin{itemize}
  \item Development of an application suite comprising of utilities assist in Data Migration.
  \item Development of cross platform desktop application using PyQt.
  \item Development of memory optimized XML parsers.
  \item Text data analysis using regular expressions.
  \end{itemize}
  \citem{K7 Computing Private Limited, Chennai} 
  \citemnobullet \textbf{Development Engineer} \hfill \textbf{Feb 2009 -- Aug 2012}
  \citemnobullet Actively involved in the design and development of a cloud computing platform 
  that can solve the problems of application scalability, availability and fault-tolerance.
  \begin{itemize} 
  \item Lead a team of 3 developers to build a scalable platform for Python \& PHP.
  \item Helped clients to migrate traditional applications to the cloud.
  \item Developed core services and Google App Engine compatible apis for a PaaS project.
  \item Developed various FUSE based userspace filesystems for linux clusters.
  \end{itemize}
\end{category}

\newpage

% -------- Awards & Honours ---------------------------------------

\begin{category}{Awards \&\\ Honours}
  \citem{``Excellence in Software Development''}, {\em K7 Computing Private Limited}. \hfill \textbf{2011}
  \citem{``Best Project''}, {\em Computer Science, Jyothi Engineering College}. \hfill \textbf{2008}
  \citem{``Best Student''}, {\em Edakkattuvayal Grama Panchayat, Ernakulam, Kerala.} \hfill \textbf{2002}
\end{category}

% -------- Publication --------------------------------------------

\begin{category}{Publication}
  \citemnobullet Technical articles published on various technology publications.
  \citem{Linux For You}, {\em ``The New Scheduler on the Block, Dedicated to Desktops''}. \hfill \textbf{Oct 2009}
  \citemnobullet An article about Brain Fuck Scheduler (BFS) for Linux written by famous Linux Kernel hacker Con Kolivas.
  \citem{ILUG-Cochin}, {\em ``Custom Kernel Compilation - Ubuntu way''}. \hfill \textbf{Aug 2008}
  \citemnobullet An article on compiling custom kernel sources on Ubuntu.
  \citem{ILUG-Cochin}, {\em ``Remastering Ubuntu''}. \hfill \textbf{Aug 2008}
  \citemnobullet An article on building custom Ubuntu distributions using Reconstructor.
\end{category}

% -------- Opensource -------------------------------------------

\begin{category}{Opensource}
  \citemnobullet List of major opensource contributions. Refer GitHub for source code.
  \citembullet \textbf{Cyclozzo}: Opensource Edition of the Cyclozzo Platform as a Service.
  \citembullet \textbf{Voldemort}: Voldemort is a blog-aware static site generator using Jinja2 
  and Markdown templates. Inspired by Jekyll, a static site generator written in Ruby.
  \citembullet \textbf{Riak}: Various fixes and feature additions to the Riak Python Client.
  \citembullet \textbf{Tuxpaint}: A magic tool for Tuxpaint. Can be used as a reference to the magic tool API.
  \citembullet \textbf{Pytt}: A simple BitTorrent tracker written in Python.
  \citembullet \textbf{GitFS}: Helps to use GitHub storage space as a filesystem under Linux.
\end{category}

% -------- Reference --------------------------------------------

\begin{category}{Reference}
  \citemnobullet \\
  \begin{tabular}{ll}Shuveb Hussain&Navin Sylvester\\
    Head, Cloud Computing&Development Manager, Cloud Computing\\
    K7 Computing Private Limited&K7 Computing Private Limited\\
    Chennai - 600041&Chennai - 600041\\
    Email: \mbox{\small\tt shuveb@gmail.com}&Email: \mbox{\small\tt navinsylvester@gmail.com}\\
    Ph: (+91)-98-4038-0386&Ph: (+91)-95-0008-7617
  \end{tabular}
\end{category}
\end{document}
